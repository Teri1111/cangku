\documentclass{article}
\usepackage[UTF8]{ctex}
\usepackage{amsmath}
\usepackage{geometry}
\geometry{a4paper,scale=0.8}
\usepackage{algorithm}
\usepackage{algpseudocode}
\usepackage{amsthm}
\usepackage{placeins} % 引入 placeins 宏包
\usepackage{fontspec}

\setmainfont{Times New Roman}

\begin{document}

\begin{center}
{\huge \textbf{Exercise 1}}

\textit{Exercises for Algorithms by Nengjun Zhu.}

\vspace{1em}
\textbf{Name:} \underline{\qquad \textbf{郭同} \qquad} \,
\textbf{Student ID:}\underline{\qquad \textbf{24721537}\qquad}\, 
\textbf{Email:}\underline{\qquad \textbf{teri@shu.edu.cn}\qquad}
\end{center}

\textbf{答案AI含量声明:} 第四题将c++代码转换为伪代码,第五题步骤e
\begin{itemize}
    


\item[1.] Suppose $a_0=1,  a_1=2, a_2=3, a_k=a_{k-1} + a_{k-2} + a_{k-3} $ for $k\geq 3$. Use strong principle of mathematical induction to prove that $a_n \leq 2^n$ for all integers $n\geq0$.

\paragraph{Proof:} When $0 \leq n \leq 4$, the conclusion is obviously true.

\paragraph{Induction Step:} For $k \geq 5$, assume $a_k \leq 2^k$, then we have:
\begin{align*}
    a_{k+1} &= a_k + a_{k-1} + a_{k-2} \\
            &\leq 2^k + 2^{k-1} + 2^{k-2} \\
            &= 2^{k-2} (4 + 2 + 1) \\
            &\leq 2^{k+1}.
\end{align*}

\paragraph{Conclusion:} Therefore, the conclusion holds.

\item[2.] {\large Consider the sorting algorithm shown in Alg.1, which is called BUBBLESORT.}
\begin{itemize}
\item[(a)] What is the minimum number of element comparisons? When is this minimum achieved?

\paragraph{Answer:} The minimum number of comparisons is $n - 1$ when the elements are already in ascending order.

\item[(b)] What is the maximum number of element comparisons? When is this maximum achieved?

\paragraph{Answer:} The maximum number of comparisons is $\frac{n(n+1)}{2}$ when the elements are in descending order.

\item[(c)] Express the running time of Alg.1 in terms of the \(O\) and \(\Omega\) notations.

\paragraph{Answer:} Let $g(n) = n - 1$, $h(n) = \frac{n(n+1)}{2}$, thus $f(n) = O(h(n))$, $f(n) = \Omega(g(n))$. The time complexity is $O(n^2)$ and $\Omega(n)$.

\item[(d)] Can the running time of the algorithm be expressed in terms of the \(\Theta\) notation? Explain.

\paragraph{Answer:} In the average case, the elements are randomly ordered, requiring about half the swaps, so the time complexity remains $\Theta(n^2)$. Both the worst-case and average-case complexities are $\Theta(n^2)$, making this notation suitable.
\end{itemize}

\item[3.] Fill in the blanks with either true (T) or false (F):

\begin{center}
    \begin{tabular}{|c|c|c|c|c|} 
    \hline
    $f(n)$ & $g(n)$ & $f=O(g) $ &$f=\Omega(g)$  &$f=\Theta(g)$ \\
    \hline\hline
    $2n^3+3n$ & $100n^2 + 2n +100$ &F &T&F \\\hline
    $50n + \log n $ & $10n+\log\log n$ &T &T& T\\\hline
    $50n\log n$ & $10n\log\log n$ & F&T& F\\\hline
    $\log n$ & $\log^2 n$ &T &F&F \\\hline
    $n!$ & $5^n$ &F  &T&F\\
    \hline
    \end{tabular}
\end{center}

\FloatBarrier % 确保浮动体不会越过这里

\item[4.] Design a divide-and-conquer algorithm to determine whether two given binary trees $T_1$ and $T_2$ are identical.

\begin{center}
\begin{algorithmic}[1]
    \State \textbf{Input:} Two binary trees, \textit{root1} and \textit{root2}
    \State \textbf{Output:} Boolean value indicating whether the trees are identical

    \Function{isSameTree}{root1, root2}
        \If{root1 = NULL and root2 = NULL}
            \State \Return true
        \EndIf
        \If{root1 = NULL or root2 = NULL}
            \State \Return false
        \EndIf
        \If{root1.val $\neq$ root2.val}
            \State \Return false
        \EndIf
        \State \Return \Call{isSameTree}{root1.left, root2.left} and \Call{isSameTree}{root1.right, root2.right}
    \EndFunction
\end{algorithmic}
\end{center}

\FloatBarrier % 确保浮动体不会越过这里

\item[5.] You are given two sorted lists of size $m$ and $n$ in ascending order. Give an $O(\log m + \log n)$ time algorithm for computing the $k$-th smallest element in the union of the two lists.

\section*{Algorithm Idea}

1. \textbf{Base Case Handling}:
   \begin{itemize}
       \item If one of the lists is empty, the $k$-th smallest element is simply the $k$-th element of the other list.
       \item If $k = 1$, return the smaller of the first elements of the two lists.
   \end{itemize}

2. \textbf{Recursive Strategy}:
   \begin{itemize}
       \item Assume the length of list $A$ is always less than or equal to the length of list $B$ (swap if necessary). This ensures the first list is never longer than the second, optimizing recursive calls.
       \item Choose $i = \min(\frac{k}{2}, m)$ and $j = \min(\frac{k}{2}, n)$ as the midpoints to split the lists $A$ and $B$.
       \item Compare $A[i-1]$ and $B[j-1]$:
       \begin{itemize}
           \item If $A[i-1] < B[j-1]$, it means that the $k$-th smallest element cannot be in the first $i$ elements of $A$. Recursively search for the $(k - i)$-th smallest element in the remaining part of $A$ and the entire $B$.
           \item Otherwise, the $k$-th smallest element cannot be in the first $j$ elements of $B$. Recursively search for the $(k - j)$-th smallest element in $A$ and the remaining part of $B$.
       \end{itemize}
   \end{itemize}

3. \textbf{Time Complexity}:
   \begin{itemize}
       \item The algorithm halves the search space in each recursive step, achieving a time complexity of $O(\log m + \log n)$.
   \end{itemize}

\section*{Detailed Steps}
\begin{enumerate}
    \item \textbf{Input:} Two sorted lists $A$ of size $m$ and $B$ of size $n$, and an integer $k$.
    \item \textbf{Output:} The $k$-th smallest element in the union of $A$ and $B$.
    \item \textbf{Procedure:}
    \begin{enumerate}
        \item If $m > n$, swap $A$ and $B$ to ensure $A$ is the smaller list.
        \item If $m = 0$, return $B[k-1]$.
        \item If $k = 1$, return $\min(A[0], B[0])$.
        \item Set $i = \min(\frac{k}{2}, m)$ and $j = \min(\frac{k}{2}, n)$.
        \item Compare $A[i-1]$ and $B[j-1]$:
        \begin{itemize}
            \item If $A[i-1] < B[j-1]$, recursively find the $(k - i)$-th smallest element in $A[i:]$ and $B$.
            \item If $A[i-1] \geq B[j-1]$, recursively find the $(k - j)$-th smallest element in $A$ and $B[j:]$.
        \end{itemize}
    \end{enumerate}
\end{enumerate}

\end{itemize}

\end{document}